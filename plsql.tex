
Antoaneta Ivanova

Software Technologies in Internet, NBU, Master Program
Course: Programming with PL/SQL


Базата данни е създадена с цел да предостави информация за спортни състезания, участниците в тях и за това кога те се провеждат и на какво място. 
Базата съдържа три таблици, първата таблица има име Activities, носи информация относно провеждането на събитията. Съдържа следните атрибути: ActivitiesID, която колона носи информация от тип Number, представлява първичен ключ за таблицата и на всяка категория спорт съответства уникален номер от тази колона. Следващия атрибут е Categories, където са описани различните категории събития, които ще бъдат спонсорирани, данните , които съдържа са от тип Text. Таблицата съдържа две колони EndDate и StartDate, които носят информация от тип Date за началната и крайната дата на всяко събитие. Колоната Type съдържа информация за това какъв е типа на провежданото събитие, а именно дали то е на закрито или открито и данните в нея са от тип Text. Колоната Location съдържа данни от тип Text и носи информация за мястото където ще се проведе събитието. 
CREATE TABLE  "ACTIVITIES" 
   (	"ACTIVITIES_ID" NUMBER(38,0) GENERATED ALWAYS AS IDENTITY MINVALUE 1 MAXVALUE 9999999999999999999999999999 INCREMENT BY 1 START WITH 1 CACHE 20 NOORDER  NOCYCLE  NOT NULL ENABLE, 
	"CATHEGORIES" VARCHAR2(225), 
	"STARTDATE" DATE, 
	"ENDDATE" DATE, 
	"TYPE" VARCHAR2(225), 
	"LOCATION" VARCHAR2(225), 
	 CONSTRAINT "ACTIVITIES_PK" PRIMARY KEY ("ACTIVITIES_ID")
  USING INDEX  ENABLE
   )
/
Втората таблица е съдържа данни за участниците във всяко събитие. Първата и колона е ID, която е първичен ключ за тази таблица, представлява уникален номер и данните са от тип Number, следващата колона е Team, данните в нея са от тип Number, и носи информация за броя на участниците във всеки отбор, следващата колона е Age, данните в нея са от тип Number и носи инфомация за възрастта на участниците във всеки отбор, която е допустима за участие в състезанията. Следващата колона е  Country, данните в нея са от тип Text и носи информация за това от коя страна е всеки отбор. 
CREATE TABLE  "PARTICIPANTS" 
   (	"ID" NUMBER(38,0) GENERATED ALWAYS AS IDENTITY MINVALUE 1 MAXVALUE 9999999999999999999999999999 INCREMENT BY 1 START WITH 1 CACHE 20 NOORDER  NOCYCLE  NOT NULL ENABLE, 
	"TEAM" NUMBER(38,0) NOT NULL ENABLE, 
	"AGE" NUMBER(38,0) NOT NULL ENABLE, 
	"COUNTRY" VARCHAR2(225) NOT NULL ENABLE, 
	 CONSTRAINT "PARTICIPANTS_PK" PRIMARY KEY ("ID")
  USING INDEX  ENABLE
   )
/

Третата таблица се нарича ACTIVITIES_DETAILS съдържа три колони- ACTN_UM, NEWLOCATION,BRANDS, колоната ACT_NUM данните в тях са от тип NUMBER и  VARCHAR . Таблиците не са свързани по между си. 
CREATE TABLE  "ACTIVITIES_DETAILS" 
   (	"ACT_NUM" NUMBER(38,0) NOT NULL ENABLE, 
	"NEWLOCATION" VARCHAR2(225), 
	"BRANDS" VARCHAR2(225), 
	 CONSTRAINT "ACTIVITIES_DETAILS_PK" PRIMARY KEY ("ACT_NUM")
  USING INDEX  ENABLE
   )
/

1. Тази функция е без параметри, връща променлива от тип номер и това, което прави е да показва коя е максималната възраст на участник в състезание. 
Таблицата върху, която се изпълнява е PARTICIPANTS, колоната е AGE. 

CREATE OR REPLACE FUNCTION maxAge
RETURN NUMBER
IS
num PARTICIPANTS.AGE%TYPE;
BEGIN 
SELECT MAX(AGE) INTO num 
FROM PARTICIPANTS;
RETURN num;
END;
Извиква се с:
BEGIN
DBMS_OUTPUT.PUT_LINE(maxAge);
END;
Резултат :
32

Statement processed.

2. Следващата процедура, показва началната дата на състезанията по зададено ID,  Изпълнява се върху таблица ACTIVITIES, и приема един входен параметър тип номер. 
create or replace procedure empshow2(eno number)
is
act_date ACTIVITIES.STARTDATE%TYPE;
BEGIN
select STARTDATE INTO act_date from ACTIVITIES where ACTIVITIES_ID = eno;
DBMS_OUTPUT.PUT_LINE(act_date);
END;

Извиква се с:
begin
empshow2(62);
end;
резултат 
12/05/2017

Statement processed.

0.05 seconds
3.Следващата процедура, показва всеки ред от таблицата, който съответства на зададен номер на ID,  Изпълнява се върху таблица ACTIVITIES, и приема един входен параметър тип номер. 
create or replace procedure empshow3(eno number)
is
act_date ACTIVITIES%ROWTYPE;
BEGIN
select * INTO act_date from ACTIVITIES where ACTIVITIES_ID = eno;DBMS_OUTPUT.PUT_LINE(act_date.CATHEGORIES||'     '|| act_date.STARTDATE||'     '|| act_date.ENDDATE||'     '|| act_date.LOCATION||'     '||act_date.TYPE);
END;
Извиква се с
begin
empshow3(62);
end;
резултат:
Skateboarding      12/05/2017     12/06/2017     Greece     Indoors

Statement processed.

0.02 seconds
4.  Следващата процедура съдържа курсор и отпечатва при извикването и данните съхранени в курсора от колоната TEAM. Няма параметри. 
CREATE OR REPLACE PROCEDURE showpart
IS
CURSOR bla IS SELECT * FROM PARTICIPANTS;
part_rec PARTICIPANTS%ROWTYPE;
BEGIN
OPEN bla;
LOOP
FETCH bla INTO part_rec;
EXIT WHEN bla%NOTFOUND;
DBMS_OUTPUT.PUT_LINE(part_rec.TEAM);
END LOOP;
CLOSE bla;
END;
BEGIN    ( извикването се изпълнява тук)
showpart;
END;

5. Процедурата showAge, приема един параметър, който е номер от ID колоната, и има една променлива , процедурата  съдържа условие,на подаденият като аргумент номер да  съответства стойност  от колоната AGE , по-малка от 32, ако това е така, тя ще бъде отпечатана. 
CREATE OR REPLACE PROCEDURE showAge(eno NUMBER)
IS
age_no PARTICIPANTS.AGE%TYPE;
BEGIN

SELECT age INTO age_no
FROM  PARTICIPANTS
WHERE ID=eno;
IF age_no < 32
THEN
DBMS_OUTPUT.PUT_LINE( age_no );
END IF;
END;
Извиква се с : 
BEGIN
showAge(1);
END;
6. Triggers Тригерът , който съм създала се изпълнява за таблица ACTIVITIES_DETAILS, като добавя нов ред, в таблицата със стойностите, които са подадени. 

CREATE OR REPLACE TRIGGER act_2
BEFORE
INSERT
ON ACTIVITIES_DETAILS
FOR EACH ROW
BEGIN
DBMS_OUTPUT.PUT_LINE('Before inserting ' || :new.BRANDS);
END;
INSERT INTO ACTIVITIES_DETAILS
VALUES('4','AMSTERDAM', 'vllm');

7. Създавам пакет с една функция. Първоначално създавам пакета и му давам име, обявявам какви функции ще има в него, само чрез техните имена и параметри , ако имат такива. След което съставям тялото на пакета, като в него е разписана функцията. Накрая декларирам променлива, в която съхранявам резултата от извикването на функцията, извикването на функцията става, чрез името на пакета и името на функцията. 
 create package part_mng as
function maxAge return NUMBER; 
end part_mng;
 
create package body part_mng as
FUNCTION maxAge
RETURN NUMBER
IS
num PARTICIPANTS.AGE%TYPE;
BEGIN 
SELECT MAX(AGE) INTO num 
FROM PARTICIPANTS;
RETURN num;
end maxAge;
end part_mng;

DECLARE
new_age NUMBER (38);
BEGIN
new_age := part_mng.maxAge;

DBMS_OUTPUT.PUT_LINE( new_age);
END;
8.  Създавам пакет, в който има една функция и една процедура, като първата е тази в задача 7 , втората е процедура е процедурата showpart, която няма параметри. При създаването на пакета , тялото и викането на фунцкията и процедурата единственото различно от описанието в горния пример е, че освен функцията имаме и процедура, която трябва да присъства като име в създаването на пакета, тялото и извикването и. 
 create package part_mng as
function maxAge return NUMBER; 
procedure showpart;
end part_mng;
 
create package body part_mng as
FUNCTION maxAge
RETURN NUMBER
IS
num PARTICIPANTS.AGE%TYPE;
BEGIN 
SELECT MAX(AGE) INTO num 
FROM PARTICIPANTS;
RETURN num;
end maxAge;

PROCEDURE showpart
IS
CURSOR bla IS SELECT * FROM PARTICIPANTS;
part_rec PARTICIPANTS%ROWTYPE;
BEGIN
OPEN bla;
LOOP
FETCH bla INTO part_rec;
EXIT WHEN bla%NOTFOUND;
DBMS_OUTPUT.PUT_LINE(part_rec.TEAM);
END LOOP;
CLOSE bla;
END showpart;
end part_mng;

DECLARE
new_age NUMBER (38);
BEGIN
part_mng.showpart;
new_age := part_mng.maxAge;
DBMS_OUTPUT.PUT_LINE( new_age);
 
END;
